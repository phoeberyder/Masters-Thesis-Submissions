\chapter{Summary and Conclusions}
\label{ch: Conclusion}

I have first outlined the steps of the method used to increase the density of Faraday Rotation measure within the Galactic plane. To do this I first increased the number of sources detected within the Galactic plane by reducing the apparent brightness of the diffuse emission in this region which was being interpreted as noise by applied a second derivative mask to the image. I then used the source finding algorithm PyBDSF to generate a catalogue of sources' positions and brightness. By comparing the positions of sources within a subsection of the image which mainly contained the Galactic plane I concluded that this method successfully increased the number of sources and source density along the Galactic plane by 200$\%$, however this was not as much an improvement as had been expected. 

The next stage in my work was to find the rotation measure values for each of the sources I had found using PyBDSF. After convolving the Stokes cubes to a consistent beam size across all frequency channels and creating a noise profile for each channel, I used the one dimensional POSSUM analysis pipeline to perform rotation measure synthesis. This pipeline extracted the spectrum for each source in the source list, transformed it from frequency to Faraday space and fit the Faraday depth function. From this the Rotation measure value of these sources was found. In this way, the increased source list allowed for an increase to the density of RM along the Galactic plane. With these RM I then constructed an RM grid to visualise the magnetic field across the image. 

I then performed a structure function analysis on this RM grid to find the average difference in RM given a separation in both position angle and angular separation between two sources. This analysis revealed that sources at all separations have a large difference, approximately 200 rad$\,$m$^{\shortminus2}$, meaning that there is rapid variation of RM across the field. This also gave an insight into what physical scale these variations were happening on, which could be as small as 1$\,$pc. It can be seen that the variations are greater along the Galactic plane, and thus future work finding the structure function of sources purely off or on the Galactic plane would further our understanding of the Galactic plane.

I also used the three dimensional POSSUM analysis pipeline to produce a cube of Faraday depth at each frequency. This pipeline also produced a two dimensional images of the peak RM and polarised intensity for the field. Much like the RM grid, these results indicated that the Galactic magnetic field was rapidly varying. They also showed interesting structure across the image, in particular filament-like depolarisation canals.

In order to check that the structure that could be seen in this maps is in fact the fractional variations in Stokes Q and U, I compared the PI and RM values found across the whole bandwidth and the values found if the RM synthesis process was run on the lowest 100 frequency channels. There was a strong positive correlation between the lower part and the full bandwidth, which means that the diffuse structure is real. 

I then investigated the possibility of multiple peaks in the rotation measure being the cause of the canals by finding each peak within the RM map. Finding the secondary and tertiary peaks within the map did not show significantly different RM values, which suggests that this explanation is not the cause of the depolarisation canals. 

Finally I conducted a gradient analysis on the RM maps for a different POSSUM field that showed significant depolarisation canals. This revealed One thing I didn't do prior to this gradient analysis is clipping of extragalactic sources. This would be a useful step for future work as it would leave a more uniform peak brightness across the field and the gradient changes should be clearer.

The capabilities of ASKAP will allow POSSUM to study and learn about the Galactic magnetic field with detail beyond any previous endeavour, with this dissertation being just a small part working towards the collaboration's aim of creating a RM grid containing one million sources. Similarly, the increased quality of data allows for work on depolarisation canals that has not previously been possible.